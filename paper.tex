%! Author = JustinMa
%! Date = 5/19/25

% Preamble
\documentclass[12pt]{article}

% Packages
\usepackage{amsmath}
\usepackage{graphicx}
\usepackage{booktabs}
\usepackage{caption}
\usepackage{float}

\title{Investigating Overfitting in Convolutional Neural Networks}
\date{\today}

% Document
\begin{document}

    \maketitle

    \section*{1. Introduction}

    Neural networks are a form of machine learning models, roughly inspired from how the human brain processes information.
    They have become increasingly prevalent in our developing world, providing powerful and unique solutions to a wide variety of problems.
    However, with power also comes risk, as training a neural network too much can cause overfitting.
    Overfitting occurs when a model learns the training data too well and performs poorly on unseen data.
    In this study, we investigate whether increasing the number of parameters in a convolutional neural network
    leads to greater overfitting, as measured by the difference between training and test accuracy.

    \section*{2. Statistical Question}

    Does increasing the number of parameters in a fixed CNN architecture cause a greater difference between training and testing accuracy?

    \subsection*{Hypotheses}

    $H_0: \beta = 0$ \newline
    $H_a: \beta \geq 0$
    \newline \newline
    Where $\beta$ is the true slope of the population least-squares regression line that relates number of parameters of the model to the difference in accuracy of the model on the train dataset and the test dataset.

    \section*{3. Data Collection}

    We trained $\mathbf{x}$ convolutional models on the CIFAR-10 dataset, each with the same architecture (pictured below) but different numbers of
    parameters ranging from approximately 1 million to 50 million.


    \section*{Data Display}


    \section*{Data Analysis}

    \section*{Conclusion}

    \section*{Reflection}
\end{document}